\section{Verstärkerschaltungen}

\subsection{Widerstandslast}
%TODO: Verstärkerschaltung aus V8S10

Problem: Ein grosser Strom ist für hohe Verstärkungen wünschenswert, kostet jedoch Spannungshub.
Die Verstärkung dieses Typs ist deshalb begrenzt auf unter 10.

\subsection{Diodenlast}
%TODO: Verstärkerschaltung aus V8S11

%TODO: Gain Formel aus V8S11

Die Spannung über die Diode nimmt gegenüber dem Strom langsamer als linear zu.
Eine Stromzuname beeinträchtigt den Ausgangsspannungsbereich nicht gleich stark wie bei der Widerstandslast.

\subsection{Stromquellenlast}
%TODO: Verstärkerschaltung aus V8S12

Über den Stromspiegel fällt lediglich ca. $V_\text{DS,sat}$ ab, das Verändern des Transistorstroms beeinträchtigt den Ausgangsspannungsbereich noch weniger.

Problem: Der Frequenzgang wird durch die Miller-Kapazität zwischen Gate und Source des Ausgangstransistors stark beeinträchtigt.

\subsection{Stromumlenkung}
%TODO: Verstärkerschaltung aus V8S13

Durch den kleinen Kleinsignalwiderstand von P1 hat die erste Verstärkerstufe eine kleine Verstärkung. 
Dadurch fällt der Miller-Effekt weniger ins Gewicht.
Dies resultiert in
\begin{itemize}
    \item verbessertem Frequenzverhalten
    \item verbessertem PSR und
    \item durch 1:Ai einstellbare, hohe Verstärkungen.
\end{itemize}
Preis dafür ist mehr Stromverbrauch sowie erhöhte Komplexität.

\subsection{Kaskode}
%TODO: Verstärkerschaltung aus V8S14

Durch Einsatz einer Kaskode bezweckt N1 keine Spannungs-, sondern eine reine Stromverstärkung, was den Miller-Effekt völlig vermeidet.
So hat auch dieser Verstärker ein gutes Frequenzverhalten.

\subsection{Wide-Swing Kaskode}
%TODO: Verstärkerschaltung (und formeln?) aus V8S15

Durch Wählen von sehr grossen $W/L$ für die Transistoren N4 und N5 wird die minimale Ausgangsspannung der Kaskude auf fast $V_\text{DS,sat}$ reduziert.

\subsection{Gefaltete Kaskode}
%TODO: Verstärkerschaltung aus V8S16

Durch „Falten“ der Kaskode kann dessen Aussteuerbereich weiter erhöht werden.
Dies führt auch zu sehr guter PSR, bedingt aber zwei Strompfade und so mehr Hardware.

\subsection{Paralleler Eingang}
%TODO: Schema (und Formel) aus V8S17

Diese Schaltung überzeugt durch
\begin{itemize}
    \item grosse Ausgangsströme und
    \item sehr grosse Spannungsverstärkung,
\end{itemize}
deren Frequenzgang leidet jedoch stark durch die Miller-Kapazität.