\section{Grundwissen}

\subsection{Bodediagramm zeichnen}
\resizebox{\columnwidth}{!}{
    \begin{tabular}{|c|c|c|}
        \hline
        \multirow{2}{*}{Element} & \multicolumn{2}{c|}{Frequenzgang} \\
        \cline{2-3}
        & Amplitudengang & Phasengang \\
        \hline
        Polstelle & $-\qty{20}{\decibel\per Dekade}$, $-\qty{3}{\decibel}$ bei der Polstelle & $-90^\circ$, $-45^\circ$ bei der Polstelle \\
        \hline
        Nullstelle & $+\qty{20}{\decibel\per Dekade}$, $+\qty{3}{\decibel}$ bei der Nullstelle & $+90^\circ$, $+45^\circ$ bei der Nullstelle \\
        \hline
    \end{tabular}
}
\medskip%

Der Anstieg / Abfall der Phase beginnt jeweils eine Dekade vor der Pol- bzw Nullstelle.

\subsection{Dezibel}
Werte in Dezibel sind immer \textbf{Leistungsverhältnisse}. 
Wird mit Spannungen gerechnet, so muss die Spannung quadriert oder der Wert in Dezibel verdoppelt werden.

\[
    g_{\qty{}{\decibel}} = 10 \cdot \log_{10} \left( \frac{P_{\rm out}}{P_{\rm in}} \right) = 20 \cdot \log_{10} \left( \frac{U_{\rm out}}{U_{\rm in}} \right)
\]

%TODO: [Flurin] Kennlinie aus V10S9
%CHECK: [Simi] @Flurin: WHY??????

