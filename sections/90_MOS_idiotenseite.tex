\section{Grundwissen}

\subsection{Bodediagramm zeichnen}
\resizebox{\columnwidth}{!}{
    \begin{tabular}{|c|c|c|}
        \hline
        \multirow{2}{*}{Element} & \multicolumn{2}{c|}{Frequenzgang} \\
        \cline{2-3}
        & Amplitudengang & Phasengang \\
        \hline
        Polstelle & $-\qty{20}{\decibel\per Dekade}$, $-\qty{3}{\decibel}$ bei der Polstelle & $-90^\circ$, $-45^\circ$ bei der Polstelle \\
        \hline
        Nullstelle & $+\qty{20}{\decibel\per Dekade}$, $+\qty{3}{\decibel}$ bei der Nullstelle & $+90^\circ$, $+45^\circ$ bei der Nullstelle \\
        \hline
    \end{tabular}
}
\medskip%

Die Phase beginnt sich jeweils eine Dekade vor der Pol- oder Nullstelle zu drehen.

\subsection{Decibel}
Werte in Decibel sind immer Leistungsverhältnisse. 
Wird mit Spannungen gerechnet, so muss die Spannung quadriert oder der Wert in Decibel verdoppelt werden.

\[
    g_{\qty{}{\decibel}} = 10 \cdot \log_{10} \frac{P_{out}}{P_{in}} = 20 \cdot \log_{10} \frac{U_{out}}{U_{in}}
\]

%TODO: Kennlinie aus V10S9