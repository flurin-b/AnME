\section{MOS Stromquelle}
Bei der Einstellung des Arbietspunkts mittels Widerstand resultiert eine quadratische Gleichung für den Strom und so die Ausgangsspannung eines Verstärkers.
Abhilfe kann eine Stromquelle anstelle des Widerstands schaffen.

MOS Transistoren sind bereits spannungsgesteuerte Stromquellen.
Durch einfügen eines $R_\text{S}$ kann der Innenwiderstand der Quelle minimiert werden.
% TODO: Schaltung mit R_S aus V6S18 und evtl. Formeln für R_iD aus V6S17

\paragraph{Minimale Ausgangsspannung}
% TODO: unterste Formel aus V6S17

\subsection{Kaskode}
Damit für die Stromquelle kein Widerstand verwendet werden muss, kann ein weiterer Transistor verwendet werden. Diese Schaltung wird Kaskode genannt.

Dabei wird der maximale Ausgangsstrom jedoch leicht reduziert.
% TODO: Kaskodenschaltung und dazugehörige Formeln aus V6S18

\paragraph{Minimale Ausgangsspannung}
% TODO: unterste zwei Formeln aus V6S18

\subsubsection{Geregelte Kaskode}
Um die Kaskodenschaltung weiter zu verbessern, kann die $V_\text{GS}$ Spannung des oberen Transistors auf die Referenzspannung geregelt werden.
Durch das Stabilisieren der Spannung wird der Arbeitspunkt des transistors stabilisiert und der Ausgangswiderstand noch grösser.
% TODO: Schaltung mit OpAMP und Formeln aus V6S20

\paragraph{Minimale Ausgangsspannung}
% TODO: unterste zwei Formeln aus V6S20

\subsubsection{Säckinger Kaskode}
Die Säckinger Kaskode ersetzt den komplexen OpAMP mit einem einzelnen Transistor in Source-Schaltung.
% TODO: Schaltung Säckinger Kaskode und Formeln aus V6S21

\paragraph{Minimale Ausgangsspannung}
% TODO: unterste zwei Formeln aus V6S21 sowie praktische Angabe aus V6S23