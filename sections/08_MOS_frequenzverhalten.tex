\section{Frequenzverhalten}

Da jede leitende Fläche eine Kapazität gegenüber der umliegenden Flächen besitzt, müssen zur Einschätzung des Frequenzgangs diverse Kapazitäten berücksichtigt werden.

\subsection{Parasitäre Kapazitäten in MOS-Transistoren}
%TODO: Evtl. Bild wie in V9S8
%TODO: Kleinsignalersatzschaltung eines Transistors inkl. Kapazitäten

\subsubsection{Nutzkapazität}
\[
    C_{oxt} = C_{ox} \cdot W_{eff} \cdot L_{eff}
\]

\subsubsection{Parasitäre Kapazitäten}
Gate-Drain- und Gate-Source-Kapazitäten durch Overlap
\[
    C_{GDt} = C_{GD} \cdot W_{eff} \cdot L_{eff}
    \qquad
    C_{GSt} = C_{GS} \cdot W_{eff} \cdot L_{eff}
\]

Source-Bulk- und Drain-Bulk-Kapazitäten durch Raumladungszone
\[
    C_{jSBt} = C_{jSB} \cdot A_S + C_{jswSB} \cdot P_S
    \qquad
    C_{jDSt} = C_{jGS} \cdot A_S + C_{jswDB} \cdot P_D
\]
$C_{jsw}$: Side Wall Kapazität
$P$: Perimiter

Kanal-Bulk-Kapazität durch Raumladungszone
\[
    C_{jBCt} = C_{jBC} \cdot W_{eff} \cdot L_{eff}
\]

Junction-Bulk-Kapazität
\[
    C_{jBC}
\]

\subsection{Vereinfachtes Handrechenmodell}

% TODO: Ersatzschaltung und Formeln aus V9S11

\begin{tabular}{lllll}
    Arbeitsbereich & & & & \\
    Gesättigt & & & & \\
    Ungesättigt & & & & \\
\end{tabular}

\subsection{Miller-Effekt}
% TODO: Bild der equivalenten Schaltungen
Das Miller Theorem postuliert, dass die linke Schaltung durch Wählen von $Y_1$ und $Y_2$ als 
\[
    Y_1(s) = Y(s) (1+A) \quad\text{und}\quad Y_2(s) = Y(s) (1+\frac{1}{A})
\]
äquivalent gemacht werden können.
Es kann durch einfaches Einsetzen bewiesen werden.

Die Miller-Kapazität $C_m$ erscheint 
\begin{itemize}
    \item multipliziert mit $1+\abs{A}$ am Eingang als $C_{mi}$ und
    \item multipliziert mit $1+\abs{\frac{1}{A}}$ am Ausgang als $C_{mo}$.
\end{itemize}

Nachteile: 
\begin{itemize}
    \item Durch Verschieben des Miller-C aus dem Vorwärtspfad stimmt die UTF nach Ersetzen des $C_m$ nicht mehr.
    \item Das Miller-Theorem geht von konstantem Frequenzgang der Verstärkung aus. Es stimmt folglich nur für die tieferen Frequenzen.
\end{itemize}

\subsubsection{Approximatives Frequenzverhalten}
Schlecht.
%TODO: Evtl. Berechnung aus V9S15-17 -> Nein, nur beschreiben, warum diese für nichts zu brauchen ist.

\subsection{Frequenzverhalten durch Zero Value Time Constant Analysis}


\subsubsection{Vorgehen}
\begin{enumerate}
    \item Kleinsignalersatzschaltung erstellen.
    \item Für alle $C_k$ die zugehörige Zeitkonstante bestimmen.
    \begin{enumerate}
        \item Alle übrigen $C_{i \neq k} = 0$ setzen.
        \item Durch Ersetzen des $C_k$ mit einer Stromquelle den von $C_k$ her gesehenen Kleinsignalwiderstand bestimmen.
        \item Zeitkonstante als $\tau_k = R_k C_k$ bzw. Polfrequenz als $f_{pk} = \frac{1}{2 \pi \tau_k}$ berechnen.
    \end{enumerate}
    \item Approximierter Frequenzgang aus DC-Verstärkung und den Polstellen zusammensetzen.
\end{enumerate}

\subsubsection{Resultate}
Das GBP ist gegeben durch den ersten Pol: 
$GPB \approx f_{p1} \cdot A_{DC}$

Die Stabilität wird bestimmt durch den zweiten Pol:
$f_{180^\circ} \approx f_{p2}$

\subsubsection{Typische Werte}
%TODO: Tabelle aus V9S23
%TODO: Typische Widerstandswerte analog V9S22
Vorsicht bei $C_{GD}$: Sollte der Transistor eine Spannungsverstärkung haben, so muss der Miller-Effekt berücksichtigt werden.

Weiter ist $C_{GD}$ bei hohen Frequenzen oft als erstes kurzgeschlossen, für den zweiten Pol muss dieser als kurzgeschlossen betrachtet werden.


