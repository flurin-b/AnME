\section{MOS Stromspiegel}
Stromspiegel werden in jeder integrierten Schaltung eingesetzt.
Sie werden dabei eingesetzt
\begin{itemize}
    \item um Arbeitspunkte einzustellen,
    \item als Eingangsstufen von OpAMPs und
    \item als grosse Lastwiderstände in Verstärkerschaltungen.
\end{itemize}

\subsection{Einfache Stromspiegel}
% TODO: Schemas p- und n-Stromspiegel wie V7S8

% TODO: Formeln aus V7S17 ergänzen
\[
    I_\text{out} = k \cdot I_\text{in} = \frac{W_\text{out}/L_\text{out}}{W_\text{in}/L_\text{in}}
\]

\[
    V_\text{out} \geq V_\text{DS, sat $\text{N}_2$} = \sqrt{\frac{2 I_\text{D}}{\mu C_\text{ox}\frac{W}{L}}}
\]

% TODO: Kleinsignalersatzschaltung Stromspiegel V7S11

\subsubsection{Eingangsbeschaltung}
Als Eingang dient ein Strom aus einer Referenzquelle
\[
    I_\text{in} = I_\text{ref}
\]
oder kann von der Eingangsspannung als
\[
    V_\text{in} = V_{\text{T, N}_1} + \sqrt{\frac{2 I_\text{in}}{\mu C_\text{ox}\frac{W}{L}}}
\]
abgeleitet werden.

\subsubsection{Anwendungen}
\paragraph{Versorgungsspannungsunterdrückung / DC-Level Shifting}
Durch Einsatz eines Stromspiegels kann die Abhängigkeit der Ausgangsspannung von der Versorgungsspannung reduziert werden.
% TODO: Bilder zur Verschiebung RL aus V7S13

\paragraph{Stromquellenlast bei Differenzstufe}
% TODO: V7S14: Leider im hintendrein, konnte keine notizen machen... :(

\paragraph{Mehrfachstromspiegel}
Durch einen Mehrfachstromspiegel reicht ein Referenzstrom aus, um diverse Referenzströme zu generieren.
Die Grösse der vom Stromspiegel erzeugten Ströme kann durch die Länge und Breite der Transistoren eingestellt werden.

\subsubsection{Optimierungen}
% TODO: Evtl. die nicht vereinfachten Formeln aus V7S19
Da $V_\text{T}$ Temperaturabhängig ist, sollten die Transistoren jeweils dieselbe Temperatur haben.

Ebenfalls ist das Teilverhältnis von $\mu C_\text{ox}$ abhängig.
Durch kontrolliertes Platzieren der Transistoren (Common Centroid Layout) und gute Temperaturkontrolle beim Herstellen können diese Werte relativ genau gehalten werden.

Zu guter Letzt sollten die $\lambda$ gleich gross sein.
Dazu müssen die Transistoren dieselbe Länge $L$ (und in einem nächsten Schritt möglichst gross) sein.


Im Grunde genommen können Stromspiegel auch in Weak und Moderate Inversion betrieben werden.
Dabei leidet jedoch die Genauigkeit.

\subsection{Wilson-Stromspiegel}
Der einfache Stromspiegel besitzt eine relativ tiefe Ausgangsimpedanz.
Abhilfe kann der Wilson-Stromspiegel schaffen.

$\text{N}_3$ bildet dabei eine Rückkopplung zur Regelung von $I_\text{o}$ auf $I_\text{i}$.
% TODO: How does this work? idgi...

% TODO: Schema p- und n-Wislon-Stromspiegel V7S21

% TODO: Formeln aus V7S21

\subsubsection{Verbesserter Wilson- / Kaskoden-Stromspiegel}

% TODO: Schem verbesserter Wilson- und Kaskoden-Stromspiegel V7S22

% TODO: Formeln zur minimalen Ausgangsspanung V7S23

\subsubsection{Stromspiegel mit geregelter Kaskode}

% TODO: Schema V7S24

Durch M4 und M5 wird die Spannung am Gate von M2 konstant gehalten. 
So wird die Ausgangsimpedanz bedeutend erhöht.

% TODO: Formeln V7S24

\subsection{Gegenüberstellung}

%TODO: Plots der verschidenen Varianten V7S24
% Evtl. bessere Darstellungen? Warscheindlich nicht lohnenswert.

% TODO: Tabelle aus V7S27
