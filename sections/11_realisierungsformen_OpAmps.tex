\section{Realisierungsformen von OpAmps}

\subsection{Einstufiger OTA}

\subsubsection{Differenzstufe mit Last}
% TODO: Schema V12S15
\begin{itemize}
    \item Meist ungenügende Verstärkung, da die Last zu klein ist.
    \item Eingangs- und Ausgangs-Common-Mode-Bereich nicht unabhängig wählbar.
\end{itemize}
\[
    a = - g_{m, N1, N2} r_{out} \qquad r_{out} = r_{N,P} \parallel r_{DS}
\]
\[
    BW = \frac{1}{2\pi r_{out} C_{L}} \qquad GBW = \frac{g_{m, N1, N2}}{2\pi C_{L}}
\]

\subsubsection{Telescopic Cascode OTA}
\begin{itemize}
    \item Geringer Stromverbrauch
    \item Hohe Verstärkung
    \item Eingangs- und Ausgangs-Common-Mode-Bereich nicht unabhängig wählbar.
    \item Kleinerer Ausgangsspannungsbereich
\end{itemize}
\[
    a = - g_{m, N1, N2} r_{out} \qquad r_{out} = r_{K\_N} \parallel r_{K\_P} \qquad r_{K\_N, K\_P} \approx r_{DS} \cdot (2+g_m r_{DS})
\]
\[
    BW = \frac{1}{2\pi r_{out} C_{L}} \qquad GBW = \frac{g_{m, N1, N2}}{2\pi C_{L}}
\]

\subsubsection{Folded Cascode OTA}
\begin{itemize}
    \item Hohe Verstärkung
    \item Eingangs- und Ausgangs-Common-Mode-Bereich unabhängig wählbar.
    \item Grosser Stromverbrauch (meist doppelt im Vergleich zu Telescopic).
\end{itemize}

\subsubsection{Symmetrischer OTA}
% TODO: Schema und Formeln aus V12S17 (Vermutlich hinter riesigem Simulationsschaltbild)
\begin{itemize}
    \item Besseres Verhalten bei hohen Frequenzkompensation, einstellbar ducrh B.
    \item Grosser Aussteuerbereich.
    \item Sehr hohe Frequenz des zweiten Pols \textrightarrow{} stabil.
\end{itemize}


\subsection{Zweistufige OTA}

\subsubsection{Zweistufiger OTA}
% TODO: Schema aus V12S20
\[
    a_V = a_{V1} \cdot a_{V2} = g_{m, N1, N2} (r_{DS_N2} \parallel r_{DS_P2}) \cdot g_{m, P3} (r_{DS_N3} \parallel r_{DS_P3} \parallel R_L)
\]
\[
    f_{p,N1} = \frac{1}{2\pi r_{N1} C_{N2}} \qquad f_{p,N3} = \frac{1}{2\pi r_{L} C_{L}}
\]

\begin{itemize}
    \item Grosse Verstärkung.
    \item $C_2 >> C_L$ nötig für Stabilität.
    \item Durch $C_C$ tiefere Bandbreite.
\end{itemize}

\subsubsection{Miller-OTA}
%TODO: Schema aus V12S22

\begin{itemize}
    \item $C_C$ nutzt den Miller-Effekt um $C_L$ zu kompensieren.
    \item Dadurch können kleinere Kapazitäten verwendet werden.
    \item Weiter werden die ersten zwei Pole auseinander geschoben. 
    \item Folglich grössere bandbreite als beim einfachen zweistufigen OTA.
\end{itemize} %TODO: überarbeiten nach V12S25

%TODO: Formeln aus V12S24&26

\subsubsection{Design-Regeln}
\begin{itemize}
    \item $\left. \abs{A(s)F(s)} \right|_{\phi = 180^\circ} < 1$
    \item $\left. \angle (A(s)F(s)) \right|_{\abs{A(s)F(s)} = 1} > 180^\circ$
    \item Zweiter Pol bei ca. $3 \cdot GBW$ wählen
\end{itemize}

