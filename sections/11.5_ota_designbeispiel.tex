\section{Designbeispiel}
\subsection{Spezifikationen}
Folgende Spezifikationen werde üblicherweise gegeben.
\begingroup
\setlength{\columnseprule}{0pt}
\begin{multicols}{2}
    \begin{description}
        \item[Open Loop Gain] $a_\text{OL}$
        \item[Last] $C_\text{L}$
        \item[Phase Margin] $\Phi_\text{M}$
        \item[Stabilität] Unity gain stable or not
        \item[Slew Rate] $SR$
        \item[Versorgungssp.] $V_\text{CC}$
        \item[Output Swing]
        \item[Offset Voltage] $V_\text{OS}$
    \end{description}
\end{multicols}
\endgroup

\subsection{Designablauf}
\begin{enumerate}
    \item Spezigikation: Definition der Ein- und Ausgänge einer Schaltung\label{enum:spec}
    \item Handrechnungen, Erstellen eines Schaltplanes
    \item Schaltkreissimulation
    \item Spezifikationen erfüllt? Ja: gut. Nein: zurück zu Schritt~\ref{enum:spec}
    \item Layout\label{enum:layout}
    \item Schaltkreissimulation mit parasitären Einflüssen
    \item Spezifikationen erfüllt? Ja: gut. Nein: zurück zu Schritt~\ref{enum:layout}
    \item Herstellen eines Prototypen\label{enum:prototyp}
    \item Test und Evaluation
    \item Spezifikationen erfüllt? Ja: gut. Nein: zurück zu Schritt~\ref{enum:prototyp}
    \item Produktion
\end{enumerate}

Wenn an irgendeinem Punkt festgestellt wird, dass die Spezifikationen nicht erreicht werden können, muss zurük zu Schritt~\ref{enum:spec} zurück gesprungen werden.

\subsubsection{Front End Design}
% TODO: Ayo im completely lost...
% In other words: I thought this could become another Betty Bossy thing. 
% It did not. We are screwed. I think on the exam (if this is even comes up) it will just be improvisation and pure chaos.
\begin{enumerate}
    \item Gegeben und Gesucht niederschreiben
    \item Grossignalanalyse: APs von Ausgang zu Eingang bestimmen.
    \begin{enumerate}[a]
        \item Sicherstellen, dass alle Transistoren gesättigt sind.
        \item Oft kommt nur Strong Inversion in Frage.
        \item Slew Rate bestimmt Biasstrom d. Ausgangsstufe: $I_\text{bias} = SR \cdot V_\text{L}$
        \item Aussteuergrenze bestimmt min. $\frac{W}{L}$ d. Ausgangsstufe: $V_\text{DS, sat} = \sqrt{2 I_\text{D} / (\mu C_\text{ox} W/L)}$
        \item Bei mehrstufigen Verstärkern: Non-Dominanter Pol bei $f_\text{nd} = 3 \cdot GBW$ wählen \textrightarrow Bestimmt $L$ der 2. Stufe.
        \item Bei mehrstufigen Verstärkern: Biasstrom der ersten Stufe mit $\frac{W}{L}$ bestimmen.
    \end{enumerate}
    \item Kleinsigalanalyse: $g_\text{m}$, $r_\text{DS}$, GBW und DC-Verstärkung bestimmen.
    \begin{enumerate}[a]
        \item $g_m = \sqrt{2I_\text{D} \cdot \mu C_\text{ox} \cdot W/L}$
        \item $r_{ds} = (a_\text{a} L + V_\text{DS}) / I_\text{D}$
        \item $GBW = a \cdot f_\text{d}$ bestimmt $g_\text{m}$ bestimmt $W/L$ des Transistors mit dominantem Pol $f_\text{d} = 1/(2\pi \cdot r_\text{out} \cdot C_\text{L})$
        \item DC-Verstärkung d.\ letzten Stufe bestimmen.
        \item $SR_1$ der ersten Stufe als $SR_1 = SR/a_2$ berechnen.
    \end{enumerate}
    \item Stabilität uns Aussteuergrenzen kontrollieren
    \item Simulation zur Kontrolle
\end{enumerate}

\paragraph{Spezielles zum Miller OTA:}

% TODO: Sorry, I missed some of this... Would probably need to be more detailed? if included at all
\[
    f_p \approx g_m / (2\pi C_L)
\]
\[
    GBW \approx g_m / (2\pi R a C_C)
\]

\paragraph{Auswirkungen einzelner Parameter:}
% TODO: Auswirkungen aus V13S22

