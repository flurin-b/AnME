\section{MOS Operationsverstärker}
Operationsverstärker ist ein Sammelbegriff für Differenzverstärker mit sehr grosser Verstärkung.

Der Ideale Operationsverstärker erfüllt zwei Bedingungen:
\begin{itemize}
    \item Es fliesst kein Strom in die Eingänge
    \item 
\end{itemize}

Man unterscheidet dabei zwischen Zwei Arten von Operationsverstärkern:
\begin{description}
    \item[OTA] Der Transimpedanz-Operationsverstärker hat eine Spannung am Eingang und liefert am Ausgang einen Strom.
    \item[OpAMP] Der OpAMP verstärkt die Eingangsspannung zu einer Ausgangsspannung
\end{description}

\subsection{Struktur}
%TODO: Bild zur Struktur V10S7
\subsubsection{Differenzstufe}
Bildet die Differenz zwischen $V+$ und $V-$ und verstärkt Differenzstufe

\subsubsection{Verstärkerstufe}
Erhöht die Verstärkung und bestimmt meist die Bandbreite.

\subsubsection{Leistungsstufe}
Wandelt die hohe Impedanz in eine kleine Ausgangsimpedanz.
Diese Stufe fehlt beim OTA.

\subsection{Differenzstufe}
\subsubsection{Grosssignalanalyse}
%TODO: Schema aus V10S8

\paragraph{Strong Inversion}
\[
    I_\text{D} = \frac{\mu V_\text{ox}}{2} \frac{W}{L} (V_\text{GS} - V_\text{T})^2
\]
\[
    I_\text{D1} + I_\text{D2} = I_\text{Q}
\]
\[
    \frac{I_\text{D}}{I_\text{Q}} = \frac{1}{2} + \frac{1}{2} \sqrt{\frac{\left(\mu C_\text{ox} \frac{W}{L}\right) \cdot V_\text{d}^2}{I_\text{Q}} - \frac{\left(\mu C_\text{ox} \frac{W}{L}\right)^2 \cdot V_\text{d}^4}{I_\text{Q}}}
\]

%TODO: Kennlinie aus V10S9

\paragraph{Weak Inversion}
\[
    I_\text{D} = \frac{W}{L} I_\text{M}' e^\frac{V_\text{GS}-V_\text{M}}{n_\text{M} V_\text{temp}}
\]

%TODO: Evtl. Formel ohne vereinfach tanh(x) = x (nicht in den Slides)

Für Kleine $V_\text{d}$:
\[
    \frac{I_\text{D}}{I_\text{Q}} \approx \frac{1}{2} \left(1+\frac{V_\text{d}}{2 n_\text{M} V_\text{temp}}\right)
\]

\paragraph{Conclusion}
Die Verstärkung ist im grossen und ganzen unabhängig von der Eingangsspannung und so vom Arbeitspunkt, der durch die Eingangsspannungen gegeben ist.

\subsubsection{Kleinsignalanalyse}

\paragraph{Ohne Stromspiegel}
% TODO: Evtl. Schema und Ersatzschaltbild aus V10S11
\[
    g_{md} = -\frac{g_m}{2}
\]

\paragraph{Mit Stromspiegel}
% TODO: Evtl. Schema und Ersatzschaltbild aus V10S11
\[
    g_{md} = -g_m
\]

% TODO: Folien V10S12 (sorry, habe nicht zugehört :( )

\subsubsection{Verstärkung}
% TODO: Jeweilige Schemas einfügen

\paragraph{Widerstandslast}

\[
    a \approx \frac{g_m (R_D \parallel R_{L\text{, ext}})}{2}
\]

\paragraph{Stromquellenlast}

\[
    a \approx \frac{ (R{L\text{, ext}} \parallel r_{DS2} \parallel r_{QL})}{2}
\]

\paragraph{Stromspiegellast}

\[
    a \approx g_m \left(\frac{1}{g_{0, N2}} \Biggm\Vert \frac{1}{g_{0, N2}} \Biggm\Vert R_{L\text{, ext}}\right)
\]

\paragraph{Grenzwertbetrachtungen}
\resizebox{\columnwidth}{!}{
    \begin{tabular}{|l|l|l|}
        \hline
        Betriebsbereich & Grenzwert der Spannungsverstärkung & Grössere Verstärkung bei \\
        \hline
        Starke Inversion & $\displaystyle \abs{a_{\text{max}}} = 2 V_e \sqrt{\frac{\mu C_\text{ox} \frac{W}{L}}{I_Q}} \approx 2 a_e \sqrt{\frac{\mu C_\text{ox} L W}{I_Q}}$ & Ruhestrom $\downarrow$, Fläche $\uparrow$, (Early-Spannung $\uparrow$) \\
        \hline
        Schwache Inversion & $\abs{a_{\text{max}}} = \frac{V_E}{n_M V_\text{temp}} \approx \frac{a_E L}{n_M V_\text{temp}}$ & (Early-Spannung $\uparrow$) \\
        \hline
    \end{tabular}
}
\medskip%

Diese Formeln sollten nicht zur Verstärkungsberechnung verwenden -- sie dienen lediglich zur Darstellung der Bezüge verschiedener Parameter.
% TODO: Evtl mehr Formeln / Bedingungen aus V10S14?