% CHECK: Lernziele
% • Sie können mindestens zwei Bereiche nennen, in
% denen Analogschaltungen heute in integrierten
% Schaltungen zur Anwendung kommen.
% • Sie kennen die Hauptschritte der Herstellung von
% integrierten Schaltungen in CMOS-Technologie.
% • Sie können die wichtigsten aktiven und passiven
% Bauelemente nennen und beschreiben, die in einem
% CMOS-Prozess standardmässig zur Verfügung
% stehen.

\section{Technologie}

\subsection{Geschichte} % CHECK: This can be removed if unneded 
\begin{description}
    \item[1926] Julius E. Lilienfeld: Erster Vorschlag zur Realisierung eines SperrschichtFET
    
    \item[1934] Oskar Heil: Erster Vorschlag eines Feldeffektverstärkers (Vorläufer vom MOSFET)
    \item[1947] W. H. Brattain, J. Bardeen (und William B.Shockley): Erfindung des ersten Bipolartransistors
    \item[1958] Jack S. Kilby: Erste Gedanken zur Realisierung einer integrierten Schaltung
    \item[1961] Robert W. Noyce: Erhält Patent für die integrierte Schaltung
    \item[1947] W. Shockley, J. Bardeen und W. Brattain: Erster funktionierender Bipolartransistor \rightarrow Physik-Nobellpreis
    \item[1958] Jack S. Kilby: Erstes IC mit 1 Transistor, \qty{17.76}{\square\milli\meter}, Realisiert RC-Oszillator
    \item[2024] Intel: Arrow Lake, >123\,Mio.Tr./mm$^2$
    \item[2024] Apple: M4max, >92\,Mio.Tr./mm$^2$
\end{description}

\paragraph{Moores Law}
1965 hat G.E. Moore in einem Paper prognostiziert, dass sich die Transistorzahl pro chip in nächsten 10 Jahren jährlich verdoppeln wird. 
1975 wurde die Prognose revidiert auf eine Verdoppelung alle zwei Jahre.

\subsection{Prozessüberblick}
Die Herstellung integrierter Schaltungen zeichnet sich durch folgende Besonderheiten aus:
\begin{itemize}
    \item Komplexe Logistik aufgrund einer Vielzahl prozessschritte
    \item Hochgradige Standardisierung
    \item Teure Infrastruktur un teure Prozesse
\end{itemize}

Der Prozess läuft in groben Zügen wie folgt ab:
\begin{enumerate}
    \item Sand wird geschmolzen und gereinigt.
    \item Daraus wird ein Silizium-Einkristall gezogen.
    \item Der Einkristall wird in Wafer geschnitten.
    \item Durch Beschichtung, Lithografie, Ätzen und Dotieren wird der Waver strukturiert.
    \item Die einzelnen Chips auf dem Waver werden vereinzelt.
    \item Zur Konfektion werden die Chips in gehäuse verbaut.
    \item Um die ICs in Systemen einzusetzen, werden diese auf Leiterplatten verbaut.
\end{enumerate}
% TODO: Image V1 S14

\paragraph{Lithographie}
Das Prinzip der Lithographie basiert auf einem lichtempfindlichen Lack, dem sogenannten Photoresist.
Dieser wird durch eine Lichtquelle löslich (positiver Photoresist) oder unlöslich (negativer Photoresist) gemacht. % CHECK: positiv / negativ korrekt?
Durch lösen des löslichen Photoresists kann die Oberfläche lokal geschützt werden und so gezielt regionen des Chips geätzt oder beschichtet werden.
Zum Ende wird der übrige Lack entfernt und der Vorgang beliebig oft wiederholt.

\paragraph{Ätzen}
Durch Ätzen kann gezielt Material von freiliegende Flächen des Wavers entfernt werden.
Dabei werden folgende Verfahren unterschieden:
\begin{description}
    \item[Isotrop (Nass oder Plasma):] Gleichförmiges Ätzen in alle Richtungen \rightarrow Bringt die Gefahr des Unterätzens
    \item[Anisotrop (Reactive Ion Etching, KOH oder Plasma):] Ätzen entlang Kristallrichtungen, z.B. KOH greift die (111)-Ebene kaum an \rightarrow Ermöglicht steiliere Gräben, MEMS
    \item[Selektiv:] Selektives Ätzen bestimmter Materialien, z.B. HF ätzt SiO$_2$ aber nicht Si \rightarrow Erlaubt das Ätzen einer Lage ohne beschädigung unterliegender Strukturen
\end{description}

\paragraph{Dotieren}
Beim Dotieren werden gezielt fremdatome in den Siliziumkristall eingepflanzt.
\begin{description}
    \item[Donatoren,] also Atome mit einem Valenzelektron mehr als der Halbleiter, verursachen einen Elektronenüberschuss, der Kristall wird n-dotiert.
    \item[Akzeptoren,] also Atome mit einem Valenzelektron weniger als der Halbleiter, verursachen einen Lochüberschuss, der Kristall wird p-dotiert.
\end{description}

\paragraph{Wafer Sort}
Die Chips werden noch als Wafer einzeln getestet. 
Dies ist oft zeitaufwändig \rightarrow durch gutes Design sollte diese Zeit minimiert werden.

Der Yield, also der prozentuale Anteil funktionaler Chips hängt dabei von der Chipgrösse ab.
Dies, da jeder Defekt bei grossen Chips eine grosse Fläche beeinträchtigt, da jeweils nur ganze Chips funktionsfähig oder defekt sein können.

Yields von \qty{90}{\percent} sind meist notwendig, um Profit zu machen.

\paragraph{Assembly and Test}
Die Waver werden in einzelne Chips getrennt und die funktionierenden Chips in gehäuse verbaut.

\subsection{Arten von Toleranzen}
Bei der Herstellung von Wavern werden verschiedene Toleranzen unterschieden:
\begin{description}
    \item[Devicetoleranz] Toleranzen betreffend der Strukturen auf einem Chip
    \item[Prozesstoleranzen] Toleranzen betreffend der Strukturen auf einem Wafer
    \item[Lostoleranz] Toleranzen innerhalb eines Batches bzw. eines Los (meist 25, selten bis 50 Wafer)
\end{description}

\subsection{CMOS Bauelemente}
Mögliche Strukturen und Elemente wie auch die Materialeigenschaften werden im Technologiehandbuch gegeben.

\subsubsection{NMOS und PMOS Transistoren}
\begin{center}
    \includesvg[scale=0.3]{images/01_CMOS.svg}
\end{center}

\subsubsection{Bipolartransistoren}
\begin{center}
    \includesvg[scale=0.3]{images/01_BJT.svg}
\end{center}

\subsubsection{Kapazitäten}
\begin{center}
    \boxed{
        C = \epsilon\frac{A}{d} = C'' \cdot A
    }
\end{center}
\begin{center}
    \boxed{
        C'' = \frac{\epsilon}{d}
    }
\end{center}
\[
    \epsilon_0 = \qty{8.85e-12}{\farad\per\meter}
\]
\[
    \epsilon_{r, \text{Si, SiO$_2$}} \approx 3.9
\]
\[
    \epsilon_{r, \text{Dielektrikum}} \approx 2.9 \text{\ (Typisch, so klein wie möglich)}
\]
\[
    C''\text{: Spezifische Kapazität}
\]

\paragraph{MIM}

Metal-Interconnect-Metal-Kondensatoren produzieren sehr kleine Kapazitäten, da die Interconnect-Layers relativ dick ($\sim \qty{2.5e-7}{\meter}$) sind und aus absichtlich ``schlechtem'' dielektrikum ($\epsilon_r \approx 2.9$) bestehen.
Die Spannungsfestigkeit ist jedoch höher.

\paragraph{MOS}

Da Oxidschichten sehr dünn realisiert werden können ($\sim \qty{2.33e-9}{\meter}$) und ein höheres $\epsilon_r \approx 3.9$ besitzen, sind diese Kondensatoren bedeutend kleiner.
Sie besitzen jedoch eine kleinere Spannungsfestigkeit.

\subsubsection{Spulen}
Spulen sind nur planar möglich und beanspruchen oft viel Platz.

\subsubsection{Widerstände}
\begin{center}
    \boxed{
        R = \rho \frac{L}{A} = \rho \frac{L}{t \cdot W} = R_\square \frac{L}{W}
    }
\end{center}
\begin{center}
    \boxed{
        R_\square = \frac{\rho}{t}
    }
\end{center}

\paragraph{Typische Werte}

\begin{tabular}{l l}
    Metall              & $R_\square \approx 0.02 ... \qty{0.08}{\ohm}$                  \\
    Poly (Salicide)     & $R_\square \approx \qty{10}{\ohm}$                             \\
    Poly (non-Salicide) & $R_\square \approx 100/\qty{400}{\ohm}$ (n+ Poly / p+ Poly)    \\
    n- / p-Diffusion    & $R_\square \approx 100/\qty{150}{\ohm}$                        \\
    n- / p-Well         & $R_\square \approx 400/\qty{1600}{\ohm}$                       \\
\end{tabular}

\subsubsection{Parasitäre Effekte}
%TODO: Bilder zu den Parasitäten V1 S27
