\section{MOSFET Grundschaltungen}
Es werden drei Grundschaltungen unterschieden. 
Diese werden jeweils durch deren Common-Anschluss benannt.

\begin{center}
    \begingroup\rowcolors{1}{white}{gray!25}
    \begin{tabular}{|c|ccc|}
        \hline
        Schaltung   & Source-Schaltung & Gate-Schaltung & Drain-Schaltung \\
        \hline
        Common      & Source    & Gate      & Drain     \\
        Eingang     & Gate      & Source    & Gate      \\
        Ausgang     & Drain     & Drain     & Source    \\
        \hline
    \end{tabular}\endgroup
\end{center}


\subsection{Einsatzgebiete und Eigenschaften}

\begin{center}
    \begingroup\rowcolors{1}{white}{gray!25}
    \begin{tabular}{|cccc|}
        \hline
        Grundschaltung          & Anwendung & $r_{in}$ & $r_{out}$ \\
        \hline
        Source                  & Tiefe -- mittlere Freq. & gross & gross \\
        Gate                    & Hohe Freq. & klein & gross \\
        Drain / Source-Folger   & Spannungsfolger, Treiber & gross & klein \\
        \hline
    \end{tabular}\endgroup
\end{center}

\subsection{Dimensionieren}
\begin{easylist}
    \ListProperties(Style1*=\bfseries,Numbers2=l,Mark1={},Mark2={)},Indent2=1em)
    @ Arbeitspunkt bestimmen
    @ Kleinsignalersatzschaltung
    @@ Beschaltung umzeichnen
    @@ Transistor durch Ersatzschaltbild ersetzen
    @ Durch lineare Analyse $a$ und $r$ berechnen
\end{easylist}

\subsection{Source-Schaltung}
% TODO: Source-Schaltung aus (oder wie in) V5S6

\subsubsection{Verstärkung}
\[
    a = \frac{v_{out}}{v_{in}} = \frac{R_D}{R_S + \frac{1}{g_m} + \frac{g_0}{g_m}(R_D + R_S)} \overset{R_G = R_S = 0}{=} - g_m (r_{ds} || R_D)
\]

\paragraph{Optimierung}
\begin{itemize}
    \item $R_S$ und $R_D$ weglassen um Chipplatz zu sparen.
    \item $R_D \to \infty$ (so gross wie möglich)
\end{itemize}

\textbf{Strong Inversion}
\[
    r_{DS} = \frac{a_E \cdot L}{I_D} \qquad g_m = \mu C_{OX} \frac{W}{L} (V_{GS} - V_T) = \frac{2 I_D}{V_{GS}-V_T}
\]
\[
    a_{max} = - \frac{g_m}{g_0} = -g_m r_{DS} = -\frac{2\cdot a_E \cdot L}{V_{GS} - V_T}
\]
\begin{itemize}
    \item $V_{GS}$ so tief wie möglich wählen ($V_{GS}-V_T \approx 150 - \qty{200}{\milli\volt}$).
    \item $L$ möglichst gross wählen.
\end{itemize}

\textbf{Weak Inversion}
\[
    g_{m} = \frac{I_D}{n_m V_{temp}} \qquad r_{DS} \approx \frac{a_E \cdot L}{I_D}
\]
\[
    a_{max} = - \frac{g_m}{g_0} = -g_m r_{DS} = -\frac{\cdot a_E \cdot L}{n_m - V_{temp}}
\]
\begin{itemize}
    \item In Weak Inversion erreicht der Transistor seine maximale Verstärkung.
    \item Sie wird durch Technologieparameter sowie $L$ bestimmt.
    \item Da mit in Weak Inversion mit Nähreungsformeln gerechnet wird, muss simuliert werden.
\end{itemize}

\subsubsection{Notizen}
\begin{itemize}
    \item Invertiert das Eingangssignal.
    \item $a_{max}$ ist der Grenzwert der Verstärkung, nicht die tatsächliche Verstärkung!
\end{itemize}

\subsection{Gate-Schaltung}
% TODO: Gate-Schaltung aus (oder wie in) V5S6

\subsubsection{Verstärkung}
\[
    a = \frac{v_{out}}{v_{in}} = \frac{R_D (1+\frac{g_0}{g_m})}{R_S + \frac{1}{g_m} + \frac{g_0}{g_m} (R_D + R_S)}
\]

\paragraph{Optimierung}
\textbf{Strong Inversion}
Für $R_S = 0$ und $R_D << 1/g_0$ gilt
\[
    a \overset{R_D \text{klein}}{\approx} g_m R_D \quad \text{bzw.} \quad a \overset{R_D \text{gross}}{\approx} \frac{g_m}{g_0} = a_{max}.
\]

%TODO: Weak inversion Grenzwerte? Oder einfach Verweis auf Source-Schaltung?

\subsubsection{Notizen}
\begin{itemize}
    \item Ohne Body-Effekt erreicht die Gate-Schaltung die gleiche Verstärkung wie die Source-Schaltung mit besserem Frequenzverhalten.
    \item Bei der Gate-Schaltung wird der Body-Effekt schnell zum Problem.
\end{itemize}

\subsection{Drain-Schaltung (Source-Follower)}
% TODO: Drain-Schaltung aus (oder wie in) V5S6

\subsubsection{Verstärkung}
\[
    a = \frac{v_{out}}{v_{in}} = \frac{R_S}{R_S + \frac{1}{g_m} + \frac{g_0}{g_m} (R_D + R_S)}
\]

\paragraph{Optimierung}
\[
    a_{max} = \lim_{R_S \to \infty} a \overset{g_m >> g_0 und r_{DS} >> R_D}{=} \lim_{R_S \to \infty} g_m \frac{R_S}{g_m R_S + 1} = 1
\]

\paragraph{Level-Shift}
Die Drain-Schaltung reduziert den DC-Pegel des Ausgangssignals um 
\[
    V_{GS} = V_T + \sqrt{\frac{2 I_D}{\mu C_{OX} \frac{W}{L}}}.
\]

\paragraph{Body Effekt}
Da die Source nicht auf Body-Potential ist, muss die Veränderung der Threshold Spannung $V_T$ aufgrund des Body-Effekts berücksichtigt werden.

\subsubsection{Notizen}
\begin{itemize}
    \item Der Source-Follower hat immer eine Verstärkung $a \leq 1$
    \item Der Source-Follower bewirkt immer einen Level-Shift um $V_{GS}$.
\end{itemize}

\subsection{Eingangs- und Ausgangswiderstände}
\begin{itemize}
    \item Fiktive Spannungsquelle ans Kleinsignalersatzschaltbild anschliessen.
    \item Strom, der über den Sorce-Knoten in den Transistor fliesst, messen.
    \item Widerstand als $r_i = \frac{u_i}{i_i}$ berechnen.
\end{itemize}

\paragraph{am Gate}
\[
    r_{i, G} \to\ \infty
\]
\paragraph{am Drain}
\[
    r_{i, S} = \frac{1}{g_m + g_0} (1 + g_0 R_D)
\]
\[
    r_{i, S} \overset{r_{DS}>>R_D}{\approx} \frac{1}{g_m + g_0}
\]
\[
    r_{i, S} \overset{g_m>>g_0}{\approx} \frac{1}{g_m}
\]
\paragraph{an Der Source}
\[
    r_{i, D} = r_{DS} (1+g_m R_S) + R_S
\]
\[
    r_{i, D} \overset{r_{DS}>>R_S}{\approx} r_{DS} (1+g_m R_S)
\]
\[
    r_{i, D} \overset{R_S = 0}{\approx} r_{DS}
\]