\section{Infrastruktur eines Chips}
% TODO: Übersicht aus V14S12

\subsection{Spannungsversorgung}
Ein wichtiger Teil der Infrastruktur ist die Spannungsversorgung. 
Die benötigte Leistung kann wie folgt abgeschätzt werden.
\[
    P_\mathrm{dynamisch} = f(C, f, N, U^2)
\]
\[
    P_\mathrm{statisch} = I_\mathrm{leakage} \cdot U^2
\]

\subsection{Arbeitspunkteinstellung}
Zur Arbeitspunkteinstellung gibt es verschiedene Strategien.
Diese können für einen der unten aufgeführten Punkte ausgelegt werden.
\begin{itemize}
    \item Konstante Spannungsamplitude
    \item Konstante Ströme
    \item Konstante Verstärkung un Transkonduktanz
\end{itemize}

\subsubsection{Verteilungsarten}
\paragraph{Voltage Mode:}

Vorteil:
\begin{itemize}
    \item Minimale Hardware
\end{itemize}
Nachteil:
\begin{itemize}
    \item Schlechtes Matching (aufgrund von Technologie und Temperaturvariationen)
    \item Störungen auf der Verbindungsleitung
\end{itemize}

\paragraph{Current Mode:}

Vorteile:
\begin{itemize}
    \item Gutes Matching (Stromspiegeltransistoren am gleichen Ort)
    \item Weniger störanfällig wegen niederohmigen Signalen
\end{itemize}
Nachteil:
\begin{itemize}
    \item Mehr Hardware
    \item Höherer Stromverbrauch
\end{itemize}

\subsection{Referenzschaltungen}
Spannungsregler sowie Arbeitspunkteinstellung und verschiedene Blöcke benötigen eine absolute Referenz.

Relative Genauigkeiten bis ca. 0.1\% ist ohne trimmen möglich, während absolute Referenzwerte Losabweichungen von bis zu 40\% aufweisen.

Entsprechend müssen absolute Referenzen trimmbar realisiert werden können.

Als Richtwert für die Qualität einer Referenz wird die Sensitivität $S$ des Referenzwerts auf die Änderung einer externen Grösse.

\subsubsection{Spannungsteiler}
$S = 1$ auf Versorgungsspannungsänderungen \textrightarrow\ unbrauchbar.

\subsubsection{Bootstrap-Referenz}
% TODO: Schema V14S16
\[
    I_D = I_M' \frac{W}{L} e^{\frac{V_{GS}-V_M}{n m V_\mathrm{temp}}}
\]
\[
    V_{GS} - V_M = n_m V_\mathrm{temp} \ln{\frac{I_D}{I_M' \frac{W}{L}}}
\]
\[
    \Delta V_{GS} = n_m V_\mathrm{temp} \ln\left(\frac{I_{D1}}{I_{D2}} \frac{(W/L)_2}{(W/L)_1}\right)
\]
\[
    I_\mathrm{bias} = \frac{\Delta V_{GS}}{R_2}
\]
$R_2$: Externer Präzisionswiderstand

Vorteile:
\begin{itemize}
    \item Unabhängig von Versorgungsspannung
\end{itemize}
Nachteile:
\begin{itemize}
    \item Proportional zur Temperatur
\end{itemize}

\textrightarrow\ Benötigt eine Start-Up Schaltung, da $I_D = \qty{0}{\ampere}$ auch stabil wäre. (nicht gezeichnet)

\subsubsection{Bandgap-Referenz}
\paragraph{Grundprinzip}
% TODO: Bild aus V14S17

Temperaturkoeffizient einer Diode: 
\[
    TC_D = \qty{-2}{\milli\volt\per\kelvin} = \frac{d V_D}{dT} \approx \frac{V_D-V_{BG}}{T}
\]

Temperaturkoeffiziend von $V_\mathrm{temp}$:
\[
    TD_{V\mathrm{temp}} = \frac{d V_\mathrm{temp}}{d T} = \frac{k}{e} \approx \qty{86.24\cdot10^{-6}}{\volt\per\kelvin} 
\]

Faktor $k$:
\[
    k = -\frac{TC_D}{TC_{V\mathrm{temp}}} = \frac{\qty{-2}{\milli\volt\per\kelvin}}{\qty{86.24\cdot10^{-6}}{\volt\per\kelvin}} = 23.2
\]

\paragraph{Realisierung}
%TODO: Schema aus V14S18

Die Bandgap-Referenz wird durch Bipolartransistoren realisert.
Die Emitter-Basis-Diode liefert dabei die Diodenspannung.
\[
    V_D = V_{EB} = V_\mathrm{temp} \ln \frac{-I_C}{I_S' A_E}
\]
%TODO: Entweder alle Formeln aus V14S18/19 oder zumindest beschreiben, wie k eingestellt wird.

Vorteile:
\begin{itemize}
    \item Unabhängig von Versorgungsspannung
    \item \qty{10}{ppm\per\kelvin} bei (\qty{0}{\degreeCelsius}-\qty{70}{\degreeCelsius})
\end{itemize}
Nachteile:
\begin{itemize}
    \item Die Spannung $V_\mathrm{ref}$ sowie der Temperaturkoeffizient hängt stark vom Offset des OpAMPs ab. \textrightarrow\ Trimmen
    \item Es ist eine Startup-Schaltung nötig.
\end{itemize}