\section{MOS Diode}
\subsection{Gegenüberstellung Diodentypen}
% TODO: Kennlinien und Schaltungen aus V6S5

\subsection{Arbeitsbereich}
Die MOS Diode arbeitet in der Sättigung, da die Bedingung für Sättigung aufgrund der Zusammengehängten Gate- und Source-Anschlüssen immer erfüllt ist:
\[
    V_\text{DS} = V_\text{GS} > V_\text{GS} - V_\text{T}
\]

Ob die Diode in Weak oder Strong Inversion arbeitet wird von der Forwardspannung bestimmt.

\subsection{Kennlinie und Kleinsignalersatzschaltung}
Die Spannung über der Diode als Funktion des Eingangsstroms und so die Diodenkennlinie wird als
\[
    V_\text{DS} = V_\text{GS} = T_\text{T} + \sqrt{\frac{2 I_\text{D}}{\mu C_\text{ox} \frac{W}{L}}}
\]
berechnet.

\[
    V_{GS} = V_\text{M} + n_m V_\text{temp} \ln{\frac{I_\text{D}}{I_m' \frac{W}{L}}}
\]

Die Kleinsignalersatzschaltung kann, wie die obige Gleichung, ebenfalls vom MOS Transistor leicht angepasst und übernommen werden.
% TODO: Ersatzschaltbild
Dabei gilt
\[
    r_\text{MD} \approx \frac{1}{g_m} = \frac{1}{\sqrt{2 \mu C_\text{ox} \frac{W}{L} I_\text{D}}}.
\]

\subsection{Anwendungen}

\subsubsection{Spannungsreferenz}
% TODO: Beide Schaltungen aus V6S10
% TODO: Create a pro and contra list type
+ Kleinerer Flächenanspruch als ein Widerstand
+ Die Eingangsspannung wird durch den relativ tiefen $\Delta r_{MD}$ geglättet
- Ebenfalls ungenau
- $r_{MD}$ kann nur schlecht verändert werden.

\subsubsection{Spannungsteiler}
% TODO: Entweder entfernen oder schemen aus V6S13 einfügen, sicher korrektes Schema einfügen
a:
 - body-Effekt bei oberem Transistor

b:
 + Da P-Transistoren kein Body-Effekt
 + Matching gut da beides P-Transistoren

c:
 - Matching schlecht, da komplementäre Elemente
 + Kein Body-Effekt

d:
 - viel Fläche
 - schlechte absolute genauigkeit
 + gute relative genauigkeit